\chapter{Question 1}
\label{intro}

\textbf{Write a Python program that extracts 1000 unique links from Twitter.  You might want to take a look at:}
\textbf{{\url{http://thomassileo.com/blog/2013/01/25/using-twitter-rest-api-v1-dot-1-with-python/}}}\\

\textbf{But there are many other similar resources available on the web.  Note that only Twitter API 1.1 is currently available; version 1 code will no longer work.\\
Also note that you need to verify that the final target URI (i.e., the one that responds with a 200) is unique.  You could have many different shortened URIs for www.cnn.com (t.co, bit.ly, goo.gl, etc.).\\
You might want to use the search feature to find URIs, or you can pull them from the feed of someone famous (e.g., Tim O'Reilly).\\
Hold on to this collection -- we'll use it later throughout the semester.}\\


For solving the above problem I used Python programming language. Following are the steps I have taken to solve the given problem:
\begin{itemize}
\item For using the Twitter API, firstly I registered for a twitter application to generate a consumer key and consumer secret.
\item Using the keys that are generated in the above step, I authenticated the application for requesting the tweets. 
\item To fetch the tweet data I started researching for packages and found multiple of them but I decided to work with `tweepy'.
\item Using `extractTweets.py', I used the tweet data that I received from the API and fetched tweet text, list of URIs in the tweet, tweet JSON and tweet id. This code is listed in Listing{\ref{lst:code1}}.
\item While extracting the data mentioned above, program broke multiple times due to overloading, to resolve this issue I wrote an exception handler to wait with a sleep time of 60*15 and then continue.
\item The data is processed in JSON format and written to a output file `tweet.json'.
\item I loaded the JSON data from the above outputted file `tweet.json', and obtained the final URI with a HTTP response code 200 by checking if the URI has any redirects in its history. 
\item To get unique URIs I stored all the final URIs obtained in the above step in a set data structure which has an inherent property of storing only unique data and have written into a file `uri.json'.
\end{itemize}

\textbf{Code Listing}
\lstinputlisting[language=Python,caption=``Python code for extracting tweets and checking for re-directs if it is a 200ok and unique then save it.'',frame=single,breaklines=true,label=lst:code1,captionpos=b,numbers=left,showspaces=false,showstringspaces=false,basicstyle=\footnotesize]{src/extractTweets.py}

