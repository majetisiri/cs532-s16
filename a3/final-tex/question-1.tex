\chapter{Question 1}
\label{intro}

\textbf{Download the 1000 URIs from assignment 2.  ``curl'', ``wget'', or ``lynx'' are all good candidate programs to use.  We want just the raw HTML, not the images, stylesheets, etc.\\
from the command line:\\
curl http://www.cnn.com/ $>$ www.cnn.com\\
wget -O www.cnn.com http://www.cnn.com/\\
lynx -source http://www.cnn.com/ $>$ www.cnn.com\\
``www.cnn.com'' is just an example output file name, keep in mind that the shell will not like some of the characters that can occur in URIs (e.g., ).  You might want to hash the URIs, like:\\
echo -n \url{``http://www.cs.odu.edu/show_features.shtml?72''} | md5
41d5f125d13b4bb554e6e31b6b591eeb\\
(``md5sum'' on some machines; note the ``-n'' in echo -- this removes the trailing newline.)\\ 
Now use a tool to remove (most) of the HTML markup.  ``lynx'' will do a fair job:}


For solving the above problem I used Python programming language. Following are the steps I have taken to solve the given problem:
\begin{itemize}
\item First I got the raw data for all the 1000 unique URIs that I collected in assignment 2, using the following  cURL command: \\
curl $<$URI$>$  $>$ $<$ output filename $>$.
\item I stored the raw HTML output generated by the cURL command in separate files for each URI and named the files based on their index. This code is listed in Listing \ref{lst:q1code1}
\item Then I got the processed HTML and stored the output in separate files for each URI using the command: \\
lynx -dump -force\textunderscore html $<$URI$>$ $>$ $<$ output filename $>$.\\
I named the files with the URI index followed by a hyphen and the word `processed'. This code is listed in Listing \ref{lst:q1code2}
\end{itemize}

\newpage
\textbf{Code Listing}
\lstinputlisting[language=Python,caption=Python code for getting raw HTML and saving them in files,frame=single,breaklines=true,label=lst:q1code1,captionpos=b,numbers=left,showspaces=false,showstringspaces=false,basicstyle=\footnotesize]{src/rawData.py}

\textbf{Code Listing}
\lstinputlisting[language=Python,caption=Python code for getting processed HTML without any images or stylesheets and saving them in files,frame=single,breaklines=true,label=lst:q1code2,captionpos=b,numbers=left,showspaces=false,showstringspaces=false,basicstyle=\footnotesize]{src/processedData.py}